\documentclass[12pt,a4paper]{article}
\usepackage{gvv}
\title{\underline{\textbf{mains-jee2}}}
\date{}
\begin{document}
\maketitle
\begin{enumerate}
  \item Considering only the principal values of the inverse trigonometric functions, the value of  
\[
    \tan \left( \sin^{-1} \left( -\frac{3}{5} \right) - 2\cos^{-1} \left( \frac{2}{\sqrt{5}} \right) \right)
\]
    is  
    \begin{itemize}
        \item[(A)] $\frac{7}{24}$
        \item[(B)] $-\frac{7}{24}$
        \item[(C)] $-\frac{5}{24}$
        \item[(D)] $\frac{5}{24}$
    \end{itemize}
		\item Let  
\[
    S = \{(x,y) \in \mathbb{R} \times \mathbb{R} \mid x \geq 0, y \geq 0, y^2 \leq 4x, y^2 \leq 12 - 2x, \text{ and } 3y + \sqrt{8}x \leq 5\sqrt{8} \}.
\]
    If the area of the region $ S $ is $ \alpha \sqrt{2} $, then $ \alpha $ is equal to  
    \begin{itemize}
        \item[(A)] $\frac{17}{2}$
        \item[(B)] $\frac{17}{3}$
        \item[(C)] $\frac{17}{4}$
        \item[(D)] $\frac{17}{5}$
    \end{itemize}
    \item Let $ k \in \mathbb{R} $. If  
\[
    \lim_{x \to 0^+} \left( \sin(\sin kx) + \cos x + x \right)^{\frac{2}{x}} = e^6,
\]
    then the value of $ k $ is  
    \begin{itemize}
        \item[(A)] 1
        \item[(B)] 2
        \item[(C)] 3
        \item[(D)] 4
    \end{itemize}
    \item Let $ f : \mathbb{R} \to \mathbb{R} $ be a function defined by  
\[
    f(x) =
    \begin{cases} 
        x^2 \sin \left( \frac{\pi}{x^2} \right), & \text{if } x \neq 0, \\
        0, & \text{if } x = 0.
    \end{cases}
\]
    Then which of the following statements is TRUE?  
    \begin{itemize}
        \item[(A)] $ f(x) = 0 $ has infinitely many solutions in the interval $ \left( \frac{1}{10^{10}}, \infty \right) $.
        \item[(B)] $ f(x) = 0 $ has no solutions in the interval $ \left( \frac{1}{\pi}, \infty \right) $.
        \item[(C)] The set of solutions of $ f(x) = 0 $ in the interval $ \left( 0, \frac{1}{10^{10}} \right) $ is finite.
        \item[(D)] $ f(x) = 0 $ has more than 25 solutions in the interval $ \left( \frac{1}{\pi^2}, \frac{1}{\pi} \right) $.
    \end{itemize}
		\item Let $ S $ be the set of all $ (\alpha, \beta) \in \mathbb{R} \times \mathbb{R} $ such that  
\[
    \lim_{x \to \infty} \frac{\sin(x^2)(\log x)^\alpha \sin\left(\frac{1}{x^2}\right)}{x^\alpha (\log(1+x))^\beta} = 0.
\]
    Then which of the following is (are) correct?
    \begin{itemize}
        \item[(A)] $ (-1, 3) \in S $
        \item[(B)] $ (-1, 1) \in S $
        \item[(C)] $ (1, -1) \in S $
        \item[(D)] $ (1, -2) \in S $
    \end{itemize}

    \item A straight line drawn from the point $ P(1, 3, 2) $, parallel to the line  
\[
    \frac{x - 2}{1} = \frac{y - 4}{2} = \frac{z - 6}{1}
\]
    intersects the plane $ L_1: x - y + 3z = 6 $ at the point $ Q $.  
    Another straight line which passes through $ Q $ and is perpendicular to the plane $ L_1 $ intersects the plane $ L_2: 2x - y + z = -4 $ at the point $ R $.  
    Then which of the following statements is (are) TRUE?
    \begin{itemize}
        \item[(A)] The length of the line segment $ PQ $ is $ \sqrt{6} $.
        \item[(B)] The coordinates of $ R $ are $ (1, 6, 3) $.
        \item[(C)] The centroid of the triangle $ PQR $ is $ \left( \frac{4}{3}, \frac{5}{3}, \frac{3}{3} \right) $.
        \item[(D)] The perimeter of the triangle $ PQR $ is $ \sqrt{2} + \sqrt{6} + \sqrt{11} $.
    \end{itemize}

    \item Let $ A_1, B_1, C_1 $ be three points in the $ xy $-plane. Suppose that the lines $ A_1C_1 $ and $ B_1C_1 $ are tangents to the curve $ y^2 = 8x $ at $ A_1 $ and $ B_1 $, respectively.  
    If $ O = (0, 0) $ and $ C_1 = (-4, 0) $, then which of the following statements is (are) TRUE?
    \begin{itemize}
        \item[(A)] The length of the line segment $ OA_1 $ is $ 4\sqrt{3} $.
        \item[(B)] The length of the line segment $ A_1B_1 $ is $ 16 $.
        \item[(C)] The orthocenter of the triangle $ A_1B_1C_1 $ is $ (0, 0) $.
        \item[(D)] The orthocenter of the triangle $ A_1B_1C_1 $ is $ (1, 0) $.
    \end{itemize}
     \item Let $ f : \mathbb{R} \to \mathbb{R} $ be a function such that $ f(x+y) = f(x) + f(y) $ for all $ x, y \in \mathbb{R} $, and $ g : \mathbb{R} \to (0, \infty) $ be a function such that $ g(x+y) = g(x)g(y) $ for all $ x, y \in \mathbb{R} $. If  
\[
    f\left( -\frac{3}{5} \right) = 12 \quad \text{and} \quad g\left( -\frac{1}{3} \right) = 2,
\]
    then the value of  
\[
    f\left( \frac{1}{4} \right) + g(-2) - 8g(0)
\]
    is \_\_\_\_.

    \item A bag contains $N$ balls out of which 3 balls are white, 6 balls are green, and the remaining balls are blue. Assume that the balls are identical otherwise. Three balls are drawn randomly one after the other without replacement. For $ i = 1,2,3 $, let $ W_i, G_i, $ and $ B_i $ denote the events that the ball drawn in the $ i^{\text{th}} $ draw is a white ball, green ball, and blue ball, respectively. If the probability  
\[
    P(W_1 \cap G_2 \cap B_3) = \frac{2}{5N}
\]
    and the conditional probability  
\[
    P(B_3 \mid W_1 \cap G_2) = \frac{2}{9},
\]
    then $ N $ equals \_\_\_\_.

    \item Let the function $ f: \mathbb{R} \to \mathbb{R} $ be defined by  
\[
    f(x) = \frac{\sin x}{e^x} \cdot \frac{x^{2023} + 2024x + 2025}{(x^2 - x + 3)}
    + \frac{x^{2023} + 2024x + 2025}{e^x (x^2 - x + 3)}.
\]
    Then the number of solutions of $ f(x) = 0 $ in $ \mathbb{R} $ is \_\_\_\_.

    \item Let $ \mathbf{p} = 2\mathbf{i} + \mathbf{j} + 3\mathbf{k} $ and $ \mathbf{q} = \mathbf{i} - \mathbf{j} + \mathbf{k} $. If for some real numbers $ \alpha, \beta, $ and $ \gamma $, we have  
\[
    15\mathbf{i} + 10\mathbf{j} + 6\mathbf{k} = \alpha (2\mathbf{p} + \mathbf{q}) + \beta (\mathbf{p} - 2\mathbf{q}) + \gamma (\mathbf{p} \times \mathbf{q}),
\]
    then the value of $ \gamma $ is \_\_\_\_.

    \item A normal with slope $ \frac{1}{\sqrt{6}} $ is drawn from the point $ (0,-\alpha) $ to the parabola $ x^2 = -4\alpha y $, where $ a > 0 $. Let $ L $ be the line passing through $ (0,-\alpha) $ and parallel to the directrix of the parabola. Suppose that $ L $ intersects the parabola at two points $ A $ and $ B $. Let $ r $ denote the length of the latus rectum and $ s $ denote the square of the length of the line segment $ AB $. If $ r:s = 1:16 $, then the value of $ 24a $ is \_\_\_\_.

    \item Let the function $ f : [1, \infty) \to \mathbb{R} $ be defined by  
\[
    f(t) =
    \begin{cases} 
    (-1)^{n+1} 2, & \text{if } t = 2n - 1, n \in \mathbb{N}, \\ 
    \frac{(2n+1)-t}{2} f(2n-1) + \frac{(t - (2n-1))}{2} f(2n+1), & \text{if } 2n - 1 < t < 2n + 1, n \in \mathbb{N}.
    \end{cases}
\]
    Define $ g(x) = \int_1^x f(t) \, dt, x \in (1, \infty) $. Let $ \alpha $ denote the number of solutions of the equation $ g(x) = 0 $ in the interval $ (1,8] $ and  
\[
    \beta = \lim\limits_{x \to 1^+} \frac{g(x)}{x-1}.
\]
    Then the value of $ \alpha + \beta $ is \_\_\_\_.

\section*{PARAGRAPH ``I"}

Let $S = \{1,2,3,4,5,6\}$ and $X$ be the set of all relations $R$ from $S$ to $S$ that satisfy both the following properties:

\begin{itemize}
    \item $R$ has exactly 6 elements.
    \item For each $(a,b) \in R$, we have $|a - b| \geq 2$.
\end{itemize}

Let $Y = \{ R \in X :$ The range of $R$ has exactly one element$\}$ and  
$Z = \{ R \in X : R$ is a function from $S$ to $S\}$.  

Let $n(A)$ denote the number of elements in a set $A$.

\item If $n(X) = \binom{m}{6}$, then the value of $m$ is \underline{\hspace{1cm}}.

\section*{PARAGRAPH ``I"}

Let $S = \{1,2,3,4,5,6\}$ and $X$ be the set of all relations $R$ from $S$ to $S$ that satisfy both the following properties:

\begin{itemize}
    \item $R$ has exactly 6 elements.
    \item For each $(a,b) \in R$, we have $|a - b| \geq 2$.
\end{itemize}

Let $Y = \{ R \in X :$ The range of $R$ has exactly one element$\}$ and  
$Z = \{ R \in X : R$ is a function from $S$ to $S\}$.  

Let $n(A)$ denote the number of elements in a set $A$.

\item If the value of $n(Y) + n(Z)$ is $k^2$, then $|k|$ is \underline{\hspace{1cm}}.

\end{enumerate}
\end{document}

