\documentclass[12pt,a4paper]{article}
\usepackage{gvv}
\title{\underline{\textbf{JEE MAINS 2024-2}}}
\date{}
\begin{document}
\maketitle
\begin{enumerate}
		\item Considering only the principal values of the inverse trigonometric functions, the value of  
\[
    \tan \brak{ \sin^{-1} \brak{ -\frac{3}{5} } - 2\cos^{-1} \brak{ \frac{2}{\sqrt{5}} } }
\]
    is  
    \begin{itemize}
        \item[(A)] $\frac{7}{24}$
        \item[(B)] $-\frac{7}{24}$
        \item[(C)] $-\frac{5}{24}$
        \item[(D)] $\frac{5}{24}$
    \end{itemize}

    \item Let $ S = \{(x,y) \in \mathbb{R} \times \mathbb{R} : x \geq 0, y \geq 0, y^2 \leq 4x, y^2 \leq 12 - 2x \text{ and } 3y + \sqrt{8}x \leq 5\sqrt{8} \} $.  
    If the area of the region $ S $ is $ \alpha \sqrt{2} $, then $ \alpha $ is equal to  
    \begin{itemize}
        \item[(A)] $\frac{17}{2}$
        \item[(B)] $\frac{17}{3}$
        \item[(C)] $\frac{17}{4}$
        \item[(D)] $\frac{17}{5}$
    \end{itemize}

    \item Let $ k \in \mathbb{R} $. If  
\[
    \lim_{x \to 0^+} \brak{ \sin(\sin kx) + \cos x + x }^{\frac{2}{x}} = e^6,
\]
    then the value of $ k $ is  
    \begin{itemize}
        \item[(A)] 1
        \item[(B)] 2
        \item[(C)] 3
        \item[(D)] 4
    \end{itemize}

    \item Let $ f : \mathbb{R} \to \mathbb{R} $ be a function defined by  
\[
    f(x) =
    \begin{cases} 
        x^2 \sin \brak{ \frac{\pi}{x^2} }, & \text{if } x \neq 0, \\
        0, & \text{if } x = 0.
    \end{cases}
\]
    Then which of the following statements is TRUE?  
    \begin{itemize}
        \item[(A)] $ f(x) = 0 $ has infinitely many solutions in the interval $ \brak{ \frac{1}{10^{10}}, \infty } $.
        \item[(B)] $ f(x) = 0 $ has no solutions in the interval $ \brak{ \frac{1}{\pi}, \infty } $.
        \item[(C)] The set of solutions of $ f(x) = 0 $ in the interval $ \brak{ 0, \frac{1}{10^{10}} } $ is finite.
        \item[(D)] $ f(x) = 0 $ has more than 25 solutions in the interval $ \brak{ \frac{1}{\pi^2}, \frac{1}{\pi} } $.
    \end{itemize}
\item Let $S$ be the set of all $(\alpha, \beta) \in \mathbb{R} \times \mathbb{R}$ such that
\[
\lim_{x \to \infty} \frac{\sin(x^2)(\log_e x)^{\alpha} \sin\bigg(\frac{1}{x^2}\bigg)}{x^{\alpha \beta} (\log_e(1+x))^{\beta}} = 0.
\]
Then which of the following is (are) correct?

\begin{itemize}
    \item[(A)] $(-1,3) \in S$
    \item[(B)] $(-1,1) \in S$
    \item[(C)] $(1,-1) \in S$
    \item[(D)] $(1,-2) \in S$
\end{itemize}

\item A straight line drawn from the point $P(1,3,2)$, parallel to the line
\[
\frac{x - 2}{1} = \frac{y - 4}{2} = \frac{z - 6}{1},
\]
intersects the plane $L_1: x - y + 3z = 6$ at the point $Q$. Another straight line which passes through $Q$ and is perpendicular to the plane $L_1$ intersects the plane $L_2: 2x - y + z = -4$ at the point $R$. Then which of the following statements is (are) TRUE?

\begin{itemize}
    \item[(A)] The length of the line segment $PQ$ is $\sqrt{6}$
    \item[(B)] The coordinates of $R$ are $(1,6,3)$
    \item[(C)] The centroid of the triangle $PQR$ is $\bigg( \frac{4}{3}, \frac{14}{3}, \frac{5}{3} \bigg)$
    \item[(D)] The perimeter of the triangle $PQR$ is $\sqrt{2} + \sqrt{6} + \sqrt{11}$
\end{itemize}

\item Let $A_1, B_1, C_1$ be three points in the $xy$-plane. Suppose that the lines $A_1C_1$ and $B_1C_1$ are tangents to the curve $y^2 = 8x$ at $A_1$ and $B_1$, respectively. If $O = (0,0)$ and $C_1 = (-4,0)$, then which of the following statements is (are) TRUE?

\begin{itemize}
    \item[(A)] The length of the line segment $OA_1$ is $4\sqrt{5}$
    \item[(B)] The length of the line segment $A_1B_1$ is $16$
    \item[(C)] The orthocenter of the triangle $A_1B_1C_1$ is $(0,0)$
    \item[(D)] The orthocenter of the triangle $A_1B_1C_1$ is $(1,0)$
\end{itemize}
\item Let $f : \mathbb{R} \to \mathbb{R}$ be a function such that $f(x+y) = f(x) + f(y)$ for all $x, y \in \mathbb{R}$, and $g : \mathbb{R} \to (0, \infty)$ be a function such that $g(x+y) = g(x) g(y)$ for all $x, y \in \mathbb{R}$. If 
\[
f\brak{-\frac{3}{5}} = 12 \quad \text{and} \quad g\brak{-\frac{1}{3}} = 2,
\]
then the value of 
\[
f\brak{\frac{1}{4}} + g(-2) - 8 g(0)
\]
is \_\_\_\_.

\item A bag contains $N$ balls out of which $3$ balls are white, $6$ balls are green, and the remaining balls are blue. Assume that the balls are identical otherwise. Three balls are drawn randomly one after the other without replacement. For $i = 1,2,3$, let $W_i, G_i,$ and $B_i$ denote the events that the ball drawn in the $i^{\text{th}}$ draw is a white ball, green ball, and blue ball, respectively. If the probability 
\[
P\brak{W_1 \cap G_2 \cap B_3} = \frac{2}{5N}
\]
and the conditional probability 
\[
P\brak{B_3 \mid W_1 \cap G_2} = \frac{2}{9},
\]
then $N$ equals \_\_\_\_.
\item Let the function $f : \mathbb{R} \to \mathbb{R}$ be defined by
\[
f(x) = \frac{\sin x}{e^{\pi x}} \cdot \frac{x^{2023} + 2024x + 2025}{x^2 - x + 3} + \frac{2}{e^{\pi x}} \cdot \frac{x^{2023} + 2024x + 2025}{x^2 - x + 3}.
\]
Then the number of solutions of $f(x) = 0$ in $\mathbb{R}$ is \_\_\_\_.

\item Let $\vec{p} = 2\hat{i} + \hat{j} + 3\hat{k}$ and $\vec{q} = \hat{i} - \hat{j} + \hat{k}$. If for some real numbers $\alpha, \beta$, and $\gamma$, we have
\[
15\hat{i} + 10\hat{j} + 6\hat{k} = \alpha \brak{2\vec{p} + \vec{q}} + \beta \brak{\vec{p} - 2\vec{q}} + \gamma \brak{\vec{p} \times \vec{q}},
\]
then the value of $\gamma$ is \_\_\_\_.
\item A normal with slope $ \frac{1}{\sqrt{6}} $ is drawn from the point $ (0,-\alpha) $ to the parabola $ x^2 = -4\alpha y $, where $ a > 0 $. Let $ L $ be the line passing through $ (0,-\alpha) $ and parallel to the directrix of the parabola. Suppose that $ L $ intersects the parabola at two points $ A $ and $ B $. Let $ r $ denote the length of the latus rectum and $ s $ denote the square of the length of the line segment $ AB $. If $ r:s = 1:16 $, then the value of $ 24a $ is \_\_\_\_.

    \item Let the function $ f : [1, \infty) \to \mathbb{R} $ be defined by  
\[
    f(t) =
    \begin{cases} 
    (-1)^{n+1} 2, & \text{if } t = 2n - 1, n \in \mathbb{N}, \\ 
    \frac{(2n+1)-t}{2} f(2n-1) + \frac{(t - (2n-1))}{2} f(2n+1), & \text{if } 2n - 1 < t < 2n + 1, n \in \mathbb{N}.
    \end{cases}
\]
    Define $ g(x) = \int_1^x f(t) \, dt, x \in (1, \infty) $. Let $ \alpha $ denote the number of solutions of the equation $ g(x) = 0 $ in the interval $ (1,8] $ and  
\[
    \beta = \lim\limits_{x \to 1^+} \frac{g(x)}{x-1}.
\]
    Then the value of $ \alpha + \beta $ is \_\_\_\_.


\section*{PARAGRAPH ``I"}

Let $S = \{1,2,3,4,5,6\}$ and $X$ be the set of all relations $R$ from $S$ to $S$ that satisfy both the following properties:

\begin{itemize}
    \item $R$ has exactly 6 elements.
    \item For each $(a,b) \in R$, we have $|a - b| \geq 2$.
\end{itemize}

Let $Y = \{ R \in X :$ The range of $R$ has exactly one element$\}$ and  
$Z = \{ R \in X : R$ is a function from $S$ to $S\}$.  

Let $n(A)$ denote the number of elements in a set $A$.

(There are two questions based on PARAGRAPH ``I", the question given below is one of them)

\item If $n(X) = \binom{m}{6}$, then the value of $m$ is \underline{\hspace{1cm}}.

\item If the value of $n(Y) + n(Z)$ is $k^2$, then $|k|$ is \underline{\hspace{1cm}}.
\section*{PARAGRAPH ``II''}

Let $ f: \brak{ 0, \frac{\pi}{2} } \to [0,1] $ be the function defined by  
\[
f(x) = \sin^2 x
\]
and let $ g: \brak{ 0, \frac{\pi}{2} } \to [0, \infty) $ be the function defined by  
\[
g(x) = \sqrt{\frac{\pi x}{2} - x^2}.
\]

(There are two questions based on PARAGRAPH ``II'', the question given below is one of them.)

\item The value of  
\[
    2 \int_{0}^{\frac{\pi}{2}} f(x) g(x)dx - \int_{0}^{\frac{\pi}{2}} g(x)dx
\]
    is \underline{\hspace{2cm}}.
\item The value of  
\[
    \frac{16}{\pi^3} \int_{0}^{\frac{\pi}{2}} f(x) g(x)dx
\]
    is \underline{\hspace{2cm}}.
\end{enumerate}
\end{document}

